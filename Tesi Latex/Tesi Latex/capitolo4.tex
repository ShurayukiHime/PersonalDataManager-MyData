\chapter{Analisi e Design}
\label{capitolo4}
\thispagestyle{empty}

\noindent Il progetto del gestore di dati personali \`e cominciato con un’analisi delle specifiche \textit{MyData} all’interno di uno studio di fattibilit\`a. Le soluzioni adottate sono generalmente di portata “enterprise” e come tali inadatte ad un progetto di tesi. Per questo motivo ho cercato di seguire, durante il progetto, le stesse linee guida e i principi posti alla base di \textit{MyData} preferendo, quando possibile, un'implementazione pi\`u semplice e adatta al contesto.

Da quanto emerso nei paragrafi precedenti, si rende necessaria la presenza dei componenti di alto livello per la realizzazione del gestore di dati personali. Questi sono:
\begin{itemize}
	\item Un utente finale che disponga di un account presso l’Operatore \textit{MyData} e di un account presso il servizio di calcolo del prossimo viaggio pi\`u probabile;
	\item Un servizio che realizzi la logica di business, ad esempio un servizio che calcola il prossimo viaggio pi\`u probabile (\textit{Most Likely Next Trip});
	\item L’Operatore \textit{MyData};
	\item Un \textit{Service Registry}, presso cui il servizio \`e registrato;
	\item Un gestore dei permessi per l’utilizzo del servizio e l’accesso ai dati dell’utente;
	\item Un Personal Data Vault per la memorizzazione dei dati.
\end{itemize}

Sono attualmente disponibili le implementazioni  di Account \cite{githubmydataaccount}, Operatore \textit{MyData} \cite{githubmydataoperator}, \textit{Service Registry} \cite{githubmydataserviceregistry} proposte dal team di sviluppo finlandese.

\section{Accounting e servizi}
Al fine di gestire gli utenti all’interno del progetto, sono stati previsti due diversi tipi di account. Il primo si ottiene per mezzo dell’iscrizione dell’utente finale al servizio \textit{MyData}. Ci\`o richiede l’inserimento obbligatorio di alcuni dati anagrafici come nome, cognome e data di nascita, che \`e possibile tuttavia modificare e arricchire in seguito. Questo tipo di account permette di associare ad ogni utente un Personal Data Vault, e costituisce punto di riferimento per tutti i singoli account creati presso i servizi. Non \`e possibile creare account duplicati.

Il secondo tipo di account fa riferimento alla registrazione che avviene al primo utilizzo di un servizio. Nel caso considerato, ad esempio, l’utente creer\`a un account specifico per il servizio \textit{Most Likely Next Trip}, associato al suo account presso \textit{MyData}; pertanto, ogni utente potr\`a avere un numero di account specifici pari ai servizi presso cui si \`e registrato. Inoltre, viene mantenuto l’elenco di tutti i permessi che l’utente ha generato per il servizio corrispondente.

Per quanto riguarda il servizio \textit{Most Likely Next Trip}, si \`e scelto di utilizzare un componente gi\`a pronto, sviluppato da Nicola Ferroni nel suo lavoro di tesi “Un sistema di previsione degli itinerari per applicazioni di smart mobility” \cite{MLNT}.  All’interno di questo progetto, si svilupper\`a un esempio di come sia possibile far interagire le due entit\`a, regolando lo scambio di dati in modo quanto pi\`u possibile aderente al modello \textit{MyData}.

\section{Operatore MyData}
\label{sec:A-mydataop}
La realizzazione di un completo \textit{MyData Operator} comporta la realizzazione di diversi componenti. Le funzioni dell’Operatore comprendono la gestione del \textit{Service Registry}, degli account utente e del processo di \textit{Service Linking}. L’Operatore deve inoltre occuparsi della reciproca identificazione fra le parti sia nel processo di autorizzazione del servizio, che durante la firma dei \textit{Consent}, e autorizzare il flusso di dati al momento della richiesta.

In questo lavoro si \`e preferito suddividere le responsabilit\`a corrispondenti a queste operazioni in una molteplicit\`a di classi, invece che un’unica entit\`a, secondo il paradigma KISS (Keep It Simple, Stupid) \cite{kissprinciple}. Pertanto si realizzeranno separatamente:
\begin{itemize}
	\item Un gestore di \textit{Consent}, che si occupa dell’emissione dei permessi e del cambio di stato degli stessi in caso di richiesta da parte dell’utente.
	\item Un \textit{Service Registry} semplificato, dove ogni servizio si registra per poter essere accessibile all’interno di \textit{MyData}.
	\item Un’entit\`a che metta in comunicazione il Personal Data Storage con il servizio, recuperando i dati necessari (punto di \textit{policy enforcement}).
	\item Un gestore per le operazioni di autenticazione fra utente e servizio.
\end{itemize}
Per questo motivo, si sceglie nel seguito del progetto di non citare pi\`u il concetto di \textit{Operator}, facendo invece riferimento alle entit\`a particolari.

\section{Service Registry, Service Linking}
Inizialmente non era stata prevista alcuna implementazione del \textit{Service Registry}, principalmente a causa della complessit\`a non solo della struttura del componente, ma anche delle operazioni di registrazione di un nuovo servizio e di \textit{Service Discovery}.

Come \`e stato descritto meglio nel capitolo \ref{capitolo5} dedicato alla Progettazione, questo componente si \`e rivelato indispensabile per il funzionamento del sistema, ma la sua complessit\`a \`e stata notevolmente ridotta. Infatti, si \`e scelto di non includere nessuna delle due descrizioni del servizio menzionate nelle specifiche, ma solo una indicazione sui tipi di dato necessari al suo funzionamento.

Questa scelta ha avuto una ripercussione anche sul protocollo di \textit{Service Linking}, che ha perso di significato in mancanza dell’entit\`a \textit{Service Registry}. Tuttavia, vista la sua importanza concettuale, ho scelto di proporne comunque una variante all’interno del meccanismo delle autorizzazioni e del \textit{Consent}.

\section{Autorizzazioni e Consent}
\subsection{Tipi di Consent}
\label{subsec:A-Consent}
Si presenta a questo punto la necessit\`a di realizzare due tipi diversi di permessi.

Il primo contiene la semantica associata al procedimento di \textit{Service Linking}: descrive quali sono le entit\`a in gioco, fornendone una mutua autenticazione e contiene uno stato per indicare la validit\`a del permesso considerato. I valori possibili di questo stato ricalcano quelli indicati nel modello \textit{MyData} (sezione \ref{sec:MD-AuthConsent}): \textit{Active}, \textit{Disabled}, \textit{Withdrawn}. Questo \textit{Consent} viene emesso ogni volta che l’utente fa richiesta di utilizzo di un servizio e, in base al valore del suo stato, permette (o meno) l’effettivo processamento dei dati.

Il secondo tipo di permesso viene utilizzato come \textit{token} di accesso al Personal Data Storage ogni volta che si instaura un flusso di dati, sia esso in ingresso o in uscita dal PDS. Questo tipo di \textit{Consent} contiene un riferimento ai tipi di dato scambiati durante la transazione (il \textit{Resource Set Identifier} delle specifiche di \textit{MyData}); inoltre, pu\`o essere utilizzato una volta sola ed \`e possibile ottenerlo solo se il permesso che lega l’utente ed il servizio corrente ha stato attivo.

Entrambi i \textit{Consent} vengono memorizzati all’interno dell’account utente corrispondente al servizio che ne ha fatto richiesta.

\subsection{Granularit\`a di un Consent}
\label{subsec:A-granularitaConsent}
Un aspetto importante da considerare in Analisi \`e la granularit\`a con cui i \textit{Consent} permettono l’accesso ai dati. 

All’interno delle specifiche di \textit{MyData} non si trovano linee guida precise riguardo questo aspetto, mentre vengono espresse delle considerazioni riguardo la frequenza di accesso ai dati e la quantit\`a di dati prelevati. L’attenzione in questo senso \`e motivata dalla volont\`a di proteggere la privacy dell’utente, che potrebbe essere messa a rischio nel caso un servizio effettuasse un numero elevato di richieste legittime per accedere a dati personali sempre diversi, acquisendo alla fine un profilo dei dati pi\`u completo e attendibile. All’interno dello sviluppo del gestore di dati personali si \`e scelto, per\`o, di privilegiare l’aspetto del controllo degli accessi piuttosto che rispondere a quanto puntualizzato in altre implementazioni \cite{githubmobilityprofile}.

In una prima ipotesi, ho preso in considerazione la possibilit\`a di garantire l’accesso ai dati in base a criteri che comprendono la tipologia, la quantit\`a di dati richiesta e la loro “et\`a”. L’implementazione di questa scelta avrebbe previsto, a grandi linee, l’utilizzo di un database per la gestione di dati con molte propriet\`a (come ad esempio la data di acquisizione sopra citata). Poich\'e tale soluzione avrebbe richiesto notevole impegno ed avrebbe sconfinato oltre l’ambito della tesi, questa possibilit\`a \`e stata scartata.

Come seconda opzione, sulla scia del Mobility Profile e della gestione delle autorizzazioni in Android, si \`e optato per un accettabile compromesso: la granularit\`a per il controllo degli accessi viene imposta in base a di tipi di dato di alto livello, come ad esempio Calendario, storico delle posizioni GPS, ecc. Pertanto, all'interno di un Consent sar\`a possibile descrivere il set di dati da trasferire soltanto attraverso indicazione del loro tipo: il controllo degli accessi si articola quindi sul confronto fra i tipi dichiarati in fase di registrazione e quelli richiesti al momento dell'emissione.

\section{Personal Data Storage}
\label{sec:A-PDS}
L’ipotesi iniziale per il Personal Data Storage era quella di realizzare un componente indipendente dalle scelte dell’utente o dalle necessit\`a del servizio, come anche dalle scelte implementative. 

Un esempio di indipendenza dalle scelte dell’utente \`e il seguente. Il servizio \textit{Most Likely Next Trip} utilizza al suo interno i dati di un calendario che contiene le coordinate GPS delle localit\`a in cui si svolgono gli eventi inseriti. La soluzione pi\`u immediata \`e quella di salvare all’interno del Personal Data Storage gli impegni necessari, o la struttura dati del calendario in un file specifico per l’applicazione di Calendario utilizzata dall’utente. Questa scelta per\`o non solo \`e poco efficiente, ma vincola all’utilizzo di un particolare tipo di calendario, impedendo successive modifiche o estensioni. Una soluzione alternativa potrebbe essere quella di utilizzare uno standard di rappresentazione di calendari, attualmente il formato \texttt{.ics}, anche se \`e ancora diffusa la versione precedente, \texttt{.vcs}. Questa scelta, seppure legata ancora ad un tipo di implementazione a file, offre maggiore interoperabilit\`a grazie alla scelta di un formato supportato anche da servizi esterni.

Alternativamente, \`e possibile astrarre ad un livello ancora maggiore l’implementazione del calendario, ad esempio utilizzando le API di servizi online (come quelle di Google Calendar \cite{googlecalendarapi}) per integrare quelli inseriti nel Personal Data Storage. Attraverso l’utilizzo di molteplici livelli di astrazione, sarebbe possibile soddisfare la richiesta dell’utente di utilizzare uno specifico calendario senza per\`o cablarne questa scelta nell’implementazione.

In ultima analisi, anche a causa di una implementazione gi\`a presente all’interno del progetto \textit{Most Likely Next Trip} si \`e scelto di adottare una persistenza che faccia uso di file di testo. Tuttavia, attraverso l’uso di interfacce si \`e cercato di non propagare le dipendenze dall’implementazione scelta all’esterno della classe, in modo da permettere un pi\`u semplice refactoring in caso di sviluppi futuri.

\section{Rappresentazione di dati non noti a priori}
\label{sec:A-datinonnotiapriori}
Una delle maggiori difficolt\`a incontrate durante lo studio di \textit{Service Registry} e Personal Data Storage riguardava la necessit\`a di descrivere tipi di dato non noti a priori, e di possedere strumenti per il loro processamento. Questa problematica interessa il \textit{Service Registry} al momento dell’iscrizione del servizio presso \textit{MyData} e il Personal Data Storage quando questo prende parte ad uno scambio di dati con il servizio.

Le specifiche del modello \textit{MyData} consentono di ottenere l’indipendenza da tipi di dato specifici tramite l’utilizzo di linguaggi che descrivono la tassonomia e le strutture dati necessarie al funzionamento del servizio. In particolare, viene fatto uso di RDF (Resource Description Framework), che permette la rappresentazione di informazioni all’interno del web \cite{w3crdf}.

Alcuni esempi utilizzati in \textit{MyData} sono W3C’s Data Catalog Vocabulary\cite{w3cdatacatalog} e The RDF Data Cube Vocabulary\cite{w3cdatacube}, utilizzati all’interno del \textit{Service Registry} nella descrizione del servizio.

Per ottenere l’indipendenza e l’interoperabilit\`a al centro del modello di \textit{MyData}, utilizzando allo stesso tempo una soluzione pi\`u adatta ad un contesto ridotto, ho considerato la possibilit\`a di trasmettere i dati necessari in linguaggio JSON.


