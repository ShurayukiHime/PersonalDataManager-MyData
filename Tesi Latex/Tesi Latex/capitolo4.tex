\chapter{Analisi e Design}
\label{capitolo4}
\thispagestyle{empty}

\noindent Il progetto del gestore di dati personali \`e cominciato con un’analisi delle specifiche \textit{MyData} all’interno di uno studio di fattibilit\`a. Le soluzioni adottate sono generalmente di portata “enterprise” e come tali inadatte ad un progetto di tesi. Per questo motivo ho cercato di seguire, durante il progetto, le stesse linee guida e i principi posti alla base di \textit{MyData} preferendo, quando possibile, una implementazione pi\`u semplice e adatta al contesto.

Da quanto emerso nei paragrafi precedenti, si rende necessaria la presenza dei componenti di alto livello per la realizzazione del gestore di dati personali. Questi sono:
\begin{itemize}
	\item Un utente finale che disponga di un account presso l’Operatore \textit{MyData} e di un account presso il servizio di calcolo del prossimo viaggio pi\`u probabile;
	\item Il servizio che realizzi la logica di business, ad esempio un servizio che calcola il prossimo viaggio pi\`u probabile (\textit{Most Likely Next Trip});
	\item L’Operatore \textit{MyData};
	\item Un \textit{Service Registry}, presso cui il servizio \`e registrato;
	\item Un gestore dei permessi per l’utilizzo del servizio e l’accesso ai dati dell’utente;
	\item Un Personal Data Vault per la memorizzazione dei dati.
\end{itemize}

\section{Accounting e servizi}
\cite{githubmydataaccount}

Al fine di gestire gli utenti all’interno del progetto, sono stati previsti due diversi tipi di account. Il primo si ottiene per mezzo dell’iscrizione dell’utente finale al servizio \textit{MyData}. Ci\`o richiede l’inserimento obbligatorio di alcuni dati anagrafici come nome, cognome e data di nascita, che \`e possibile tuttavia modificare e arricchire in seguito. Questo tipo di account permette di associare ad ogni utente un Personal Data Vault, e costituisce punto di riferimento per tutti i singoli account creati presso i servizi. Non \`e possibile creare account duplicati.

Il secondo tipo di account fa riferimento alla registrazione che avviene al primo utilizzo di un servizio. Nel caso considerato, ad esempio, l’utente creer\`a un account specifico per il servizio Most Likely Next Trip, associato al suo account presso \textit{MyData}; pertanto, ogni utente potr\`a avere un numero di account specifici pari ai servizi presso cui si \`e registrato. Inoltre, viene mantenuto l’elenco di tutti i permessi che l’utente ha generato per il servizio corrispondente.

Per quanto riguarda il servizio Most Likely Next Trip, si \`e scelto di utilizzare un componente gi\`a pronto, sviluppato da Nicola Ferroni nel suo lavoro di tesi “Un sistema di previsione degli itinerari per applicazioni di smart mobility”\cite{MLNT}.  All’interno di questo progetto, si svilupper\`a un esempio di come sia possibile far interagire le due entit\`a, regolando lo scambio di dati in modo quanto pi\`u possibile aderente al modello \textit{MyData}.

\section{Operatore MyData}
\cite{githubmydataoperator}
La realizzazione di un completo \textit{MyData} Operator comporta la costruzione di diversi componenti. Le funzioni dell’Operatore comprendono la gestione del \textit{Service Registry}, degli account utente e del processo di \textit{Service Linking}. L’Operatore deve inoltre occuparsi della reciproca identificazione fra le parti nel processo di autorizzazione del servizio, durante la firma dei \textit{Consent}, e autorizzare il flusso di dati quando ne viene fatta richiesta.

In questo lavoro si \`e preferito suddividere le responsabilit\`a corrispondenti a queste operazioni in una molteplicit\`a di classi, invece che un’unica entit\`a, in conformit\`a a principi di semplicit\`a come il rasoio di Occam o l’acronimo KISS (Keep It Simple, Stupid). Pertanto, si realizzeranno separatamente:
\begin{itemize}
	\item Un gestore di \textit{Consent}, che si occupi dell’emissione dei permessi e del cambio di stato degli stessi in caso di richiesta da parte dell’utente.
	\item Un \textit{Service Registry} semplificato, dove ogni servizio si registra per poter essere accessibile all’interno di \textit{MyData}.
	\item Un’entit\`a che metta in comunicazione il Personal Data Storage con il servizio, recuperando i dati necessari (punto di \textit{policy enforcement}).
	\item Un gestore per le operazioni di autenticazione fra utente e servizio.
\end{itemize}
Per questo motivo, si sceglie nel seguito del progetto di non citare pi\`u il concetto di Operator, facendo invece riferimento alle entit\`a particolari.

\section{Service Registry, Service Linking}
Una implementazione proposta dal team di sviluppo finlandese pu\`o essere trovata all’indirizzo \cite{githubmydataserviceregistry}. 

Inizialmente non era stata prevista alcuna implementazione di \textit{Service Registry}, principalmente a causa della complessit\`a non solo della struttura del componente, ma anche delle operazioni di registrazione di un nuovo servizio e di \textit{Service Discovery}.

Come verr\`a descritto meglio nella sezione dedicata alla Progettazione, questo componente si \`e rivelato indispensabile per il funzionamento del sistema, ma la sua complessit\`a \`e stata notevolmente ridotta. Infatti, si \`e scelto di non includere nessuna delle due descrizioni del servizio menzionate nelle specifiche, ma solo una indicazione sui tipi di dato necessari al suo funzionamento.

Questa scelta ha avuto una ripercussione anche sul protocollo di \textit{Service Linking}, che ha perso di significato in mancanza dell’entit\`a \textit{Service Registry}. Tuttavia, vista la sua importanza concettuale, ho scelto di proporne comunque una variante all’interno del meccanismo delle autorizzazioni e del \textit{Consent}.

\section{Consent}
\label{sec:A-Consent}
A questo punto, si presenta la necessit\`a di realizzare due tipi diversi di permessi.

Il primo contiene la semantica associata al procedimento di \textit{Service Linking}: descrive quali sono le entit\`a in gioco, fornendone una mutua autenticazione e contiene uno stato per indicare la validit\`a del permesso considerato. I valori possibili di questo stato ricalcano quelli indicati nel modello \textit{MyData}: Active, Disabled, Withdrawn. Questo \textit{Consent} viene emesso ogni volta che l’utente fa richiesta di utilizzare un servizio, e in base al valore del suo stato permette (o meno) l’effettivo processamento dei dati.

Il secondo tipo di permesso viene utilizzato come token di accesso al Personal Data Storage ogni volta che si instaura un flusso di dati, sia esso in ingresso o in uscita dal PDS. Questo tipo di \textit{Consent} contiene un riferimento ai tipi di dato scambiati durante la transazione, in modo da poter verificare se essi coincidano con quelli dichiarati dal servizio in fase di registrazione. Inoltre, pu\`o essere utilizzato una volta sola ed \`e possibile ottenerlo solo se il permesso che lega l’utente ed il servizio corrente ha stato attivo.

Entrambi i \textit{Consent} vengono memorizzati all’interno dell’account utente corrispondente al servizio che ne ha fatto richiesta.

\section{Personal Data Storage}
L’ipotesi iniziale per il Personal Data Storage era quella di realizzare un componente indipendente dalle scelte dell’utente o dalle necessit\`a del servizio, come anche dalle scelte implementative. 

Un esempio di indipendenza dalle scelte dell’utente \`e il seguente. Il servizio Most Likely Next Trip utilizza al suo interno i dati di un calendario, secondo l’assunto per cui l’utente deve compiere un viaggio per portarsi nel luogo dell’evento inserito. La soluzione pi\`u immediata \`e quella di salvare all’interno del PDS gli impegni necessari, o la struttura dati del calendario in un file specifico per l’applicazione di Calendario utilizzata dall’utente. Questa scelta per\`o non solo \`e poco efficiente, ma vincola all’utilizzo di un particolare tipo di calendario, impedendo successive modifiche o estensioni. Una soluzione alternativa potrebbe essere quella di utilizzare uno standard di rappresentazione di calendari, attualmente il formato \texttt{.ics}, anche se \`e ancora diffusa la versione precedente, \texttt{.vcs}. Questa scelta, seppure legata ancora ad un tipo di implementazione a file, offre maggiore interoperabilit\`a grazie alla scelta di un formato supportato anche da servizi esterni.

Alternativamente, \`e possibile astrarre ad un livello ancora maggiore l’implementazione del calendario, ad esempio utilizzando le API di servizi online (come quelle di Google Calendar\cite{googlecalendarapi}) per integrare quelli inseriti nel Personal Data Storage. Attraverso l’utilizzo di molteplici livelli di astrazione sarebbe possibile soddisfare la richiesta dell’utente di utilizzare uno specifico calendario, senza per\`o cablarne questa scelta nell’implementazione.

In ultima analisi, anche a causa di una implementazione gi\`a presente all’interno del progetto Most Likely Next Trip si \`e scelto di adottare una persistenza che faccia uso di file di testo. Tuttavia, attraverso l’uso di interfacce ho cercato di non propagare le dipendenze dall’implementazione scelta all’esterno della classe, in modo da permettere un pi\`u semplice refactoring in caso di sviluppi futuri.

\section{Rappresentazione di dati non noti a priori}
Una delle maggiori difficolt\`a incontrate durante lo studio di \textit{Service Registry} e Personal Data Storage riguardava la necessit\`a di descrivere tipi di dato non noti a priori, e di possedere strumenti per il loro processamento. Questa problematica interessa il \textit{Service Registry} al momento dell’iscrizione del servizio presso \textit{MyData} e il Personal Data Storage quando questo prende parte ad uno scambio di dati con il servizio.

Le specifiche del modello \textit{MyData} consentono di ottenere l’indipendenza da tipi di dato specifici tramite l’utilizzo linguaggi che descrivono la tassonomia e le strutture dati necessarie al funzionamento del servizio. In particolare, viene fatto uso di RDF (Resource Description Framework), che permette la rappresentazione di informazioni all’interno del web \cite{w3crdf}.

Alcuni esempi utilizzati in \textit{MyData} sono W3C’s Data Catalog Vocabulary\cite{w3cdatacatalog} e The RDF Data Cube Vocabulary\cite{w3cdatacube}, utilizzati all’interno del \textit{Service Registry} nella descrizione del servizio.

Per ottenere l’indipendenza e l’interoperabilit\`a al centro del modello di \textit{MyData}, utilizzando allo stesso tempo una soluzione pi\`u adatta ad un contesto ridotto, ho preso considerato la possibilit\`a di trasmettere i dati necessari in linguaggio JSON.


