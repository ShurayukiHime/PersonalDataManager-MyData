\chapter{Introduzione}
\label{Introduzione}
\thispagestyle{empty}

\noindent Si provi a pensare alla quantit\`a di informazioni che uno smartphone raccoglie ogni giorno sul suo proprietario. Dati di localizzazione insieme a statistiche di utilizzo possono chiarire in quale parte del mondo ci troviamo e, ad esempio, che tipologia di occupazione abbiamo. Applicazioni di social networking tengono conto di etnia, orientamento politico, sessuale, religioso, insieme alla rete di amici e conoscenti acquisita dalla lista dei contatti. Calendario ed email possono aiutare in caso di impegni di lavoro; ci sono poi app per il monitoraggio dell’attivit\`a fisica che contengono grandi quantit\`a di informazioni biometriche. Per concludere non si pu\`o dimenticare la presenza di app di e-commerce, gestione di account bancari e genericamente finanziarie. Il tacito accordo che porta avanti questa relazione fra gli utenti e le aziende che erogano servizi accessibili via Internet consiste nella rinuncia da parte dell’utilizzatore del controllo sui propri dati e sull’uso che ne viene fatto in cambio di servizi gratuiti o a basso prezzo. La quantit\`a di informazioni raccolte non si limita a ci\`o che l’utente sceglie di condividere (foto, email, transazioni online), ma comprende anche dati osservati (abitudini di navigazione, dati di geolocalizzazione) e deduzioni (targeted advertising, previsioni sul flusso del traffico) \cite{IAFmDavis}.

Tuttavia, in questo contesto \`e in crescita una contro-tendenza che vede un numero sempre pi\`u crescente di utenti insoddisfatti dell’uso dei propri dati personali \cite{orangeDigitalTrust}, \cite{wefreport}. L’esplorazione in ambito scientifico di soluzioni che offrono una maggiore tutela della privacy si affianca ad una crescente attenzione da parte dell’opinione pubblica e delle istituzioni (adozione del regolamento generale sulla protezione dei dati da parte dell’Unione Europea \cite{gdpr}).

In questo contesto si sviluppa il modello \textit{MyData}, fondato sull’idea secondo cui ogni utente ha il controllo sui propri dati personali ed \`e a conoscenza dell’utilizzo che ne viene fatto, in modo trasparente. D’altra parte, l’architettura \textit{MyData} vuole offrire al mercato e alle imprese un contesto di sviluppo di applicazioni software in cui vengono rispettate le leggi in materia di protezione dei dati sensibili, favorendo l’interoperabilit\`a fra di esse.

L’approccio \textit{MyData}, seppur innovativo, non \`e senza precedenti. Molti dei suoi concetti fondamentali sono anche alla base dei personal cloud, come ad esempio la possibilit\`a di contenere in modo sicuro i dati personali dell’utente e offrire interoperabilit\`a fra diversi servizi, utilizzando concretamente i dati memorizzati.

Questa tesi si propone di studiare i concetti fondamentali alla base del modello \textit{MyData}, e implementare un prototipo di gestore di dati personali che ne rispetti le specifiche.

A titolo di esempio, si \`e scelto di utilizzare come servizio “consumatore” di dati personali un sistema di previsione del viaggio pi\`u probabile.

L’obiettivo \`e quello di realizzare un sistema che permetta all’utente di monitorare l’utilizzo dei suoi dati personali in tempo reale ed in modo trasparente, attraverso l’implementazione di una politica di controllo degli accessi. Inoltre, il gestore opera in un contesto in cui il rispetto delle politiche di sicurezza \`e fondamentale: anche se la realizzazione di un sistema sicuro non rientra negli obiettivi del lavoro di tesi, sono presenti alcuni accorgimenti che puntano in questa direzione.

\section{MyData, Big Data}
Attualmente, grandi quantit\`a di dati vengono continuamente raccolti senza utilizzarne pienamente la ricchezza dell’informazione. Il loro percorso e le operazioni di processamento a cui vengono sottoposti sono sempre pi\`u numerose e cresce la difficolt\`a nel conservarne le tracce. Inoltre, l’utilizzo di software proprietari limita la possibilit\`a di analisi dei dati a causa della scarsa interoperabilit\`a fra le soluzioni adottate. Infine, va anche considerato l’impatto che tale modus operandi ha sugli utilizzatori finali: secondo un sondaggio riportato dal World Economic Forum, 
\begin{quote}
	“Fully 78\% of consumers think it is hard to trust companies when it comes to use of their personal data” \cite{orangeDigitalTrust}.
\end{quote}

Il progetto \textit{MyData} nasce in Finlandia, all’universit\`a di Aalto, dall’idea di rafforzare i diritti digitali dell’individuo, e di offrire un modello per la gestione di dati personali che aderisca alla stringente legislazione europea sulla loro tutela. Inoltre, \textit{MyData} cerca di trovare una risposta a questi temi attraverso un cambio di paradigma che ponga l’utente al centro, attribuendogli la capacit\`a di gestire i propri dati personali e di comprendere l’uso che ne viene fatto. Ulteriori benefici includerebbero l’aumento della qualit\`a del servizio ricevuto, grazie ad una pi\`u consapevole condivisione (poich\'e informata) dei propri dati. Scenari futuri includono la possibilit\`a di fornire conoscenze pi\`u approfondite del proprio comportamento da utenti (self tracking), anche attraverso la ricezione di un compenso per il processamento dei propri dati (data monetization) \cite{mydatawhitepaper}.

Vi sono in aggiunta benefici anche in termini di sviluppo software. L’approccio a livello infrastrutturale permetterebbe l’indipendenza da specifici settori (sanit\`a e salute, finanza, ecc.), favorendo una completa portabilit\`a dei dati. Il rispetto e l’aderenza alle leggi verrebbe realizzato dall’infrastruttura stessa, consentendo alle aziende di sviluppare i servizi informatici in maniera meno vincolata, acquisendo al contempo la fiducia del cliente grazie alla trasparenza e all’affidabilit\`a garantite da \textit{MyData}.

Nel complesso, il paradigma \textit{MyData} si contrappone all’attuale standard, Big Data, proponendo nuove soluzioni che non trascurino la realt\`a del mercato in cui il software viene usato.

\section{Struttura della tesi}

In questa sezione vengono sinteticamente illustrati i criteri utilizzati nello svolgimento del lavoro.

Nel capitolo \ref{capitolo2} contiene il contesto nel quale si sviluppa il modello e la necessit\`a di un gestore di dati personali. Viene descritto un esempio di implementazione dei principi di \textit{MyData} nell'ambito Smart Mobility.

Il capitolo \ref{capitolo3} contiene l'archiettura di \textit{MyData} ed i principi che sono posti alla base e costituiscono il punto di partenza considerato per lo studio di fattibilit\`a e l'implementazione del progetto. All'interno delle diverse sezioni si considerano ad alto livello le soluzioni implementative proposte dal team di sviluppo \textit{MyData}.

Il capitolo \ref{capitolo4} \`e dedicato all'analisi del problema. Sono individuate le problematiche poste dalla realizzazione del gestore di dati personali e le soluzioni pi\`u adatte, mettendo in evidenza le entit\`a in gioco e le relazioni fra di esse.

La fase di realizzazione viene trattata nel capitolo \ref{capitolo5}. Qui vengono mostrate concretamente le scelte implementative derivate dalle considerazioni svolte nei capitoli precedenti.

Infine, nel capitolo \ref{capitolo6} si presentano le conclusioni e i possibili futuri sviluppi del lavoro di tesi.

